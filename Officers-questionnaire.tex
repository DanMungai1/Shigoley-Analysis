% Options for packages loaded elsewhere
\PassOptionsToPackage{unicode}{hyperref}
\PassOptionsToPackage{hyphens}{url}
\PassOptionsToPackage{dvipsnames,svgnames,x11names}{xcolor}
%
\documentclass[
  letterpaper,
  DIV=11,
  numbers=noendperiod]{scrartcl}

\usepackage{amsmath,amssymb}
\usepackage{iftex}
\ifPDFTeX
  \usepackage[T1]{fontenc}
  \usepackage[utf8]{inputenc}
  \usepackage{textcomp} % provide euro and other symbols
\else % if luatex or xetex
  \usepackage{unicode-math}
  \defaultfontfeatures{Scale=MatchLowercase}
  \defaultfontfeatures[\rmfamily]{Ligatures=TeX,Scale=1}
\fi
\usepackage{lmodern}
\ifPDFTeX\else  
    % xetex/luatex font selection
\fi
% Use upquote if available, for straight quotes in verbatim environments
\IfFileExists{upquote.sty}{\usepackage{upquote}}{}
\IfFileExists{microtype.sty}{% use microtype if available
  \usepackage[]{microtype}
  \UseMicrotypeSet[protrusion]{basicmath} % disable protrusion for tt fonts
}{}
\makeatletter
\@ifundefined{KOMAClassName}{% if non-KOMA class
  \IfFileExists{parskip.sty}{%
    \usepackage{parskip}
  }{% else
    \setlength{\parindent}{0pt}
    \setlength{\parskip}{6pt plus 2pt minus 1pt}}
}{% if KOMA class
  \KOMAoptions{parskip=half}}
\makeatother
\usepackage{xcolor}
\setlength{\emergencystretch}{3em} % prevent overfull lines
\setcounter{secnumdepth}{-\maxdimen} % remove section numbering
% Make \paragraph and \subparagraph free-standing
\ifx\paragraph\undefined\else
  \let\oldparagraph\paragraph
  \renewcommand{\paragraph}[1]{\oldparagraph{#1}\mbox{}}
\fi
\ifx\subparagraph\undefined\else
  \let\oldsubparagraph\subparagraph
  \renewcommand{\subparagraph}[1]{\oldsubparagraph{#1}\mbox{}}
\fi


\providecommand{\tightlist}{%
  \setlength{\itemsep}{0pt}\setlength{\parskip}{0pt}}\usepackage{longtable,booktabs,array}
\usepackage{calc} % for calculating minipage widths
% Correct order of tables after \paragraph or \subparagraph
\usepackage{etoolbox}
\makeatletter
\patchcmd\longtable{\par}{\if@noskipsec\mbox{}\fi\par}{}{}
\makeatother
% Allow footnotes in longtable head/foot
\IfFileExists{footnotehyper.sty}{\usepackage{footnotehyper}}{\usepackage{footnote}}
\makesavenoteenv{longtable}
\usepackage{graphicx}
\makeatletter
\def\maxwidth{\ifdim\Gin@nat@width>\linewidth\linewidth\else\Gin@nat@width\fi}
\def\maxheight{\ifdim\Gin@nat@height>\textheight\textheight\else\Gin@nat@height\fi}
\makeatother
% Scale images if necessary, so that they will not overflow the page
% margins by default, and it is still possible to overwrite the defaults
% using explicit options in \includegraphics[width, height, ...]{}
\setkeys{Gin}{width=\maxwidth,height=\maxheight,keepaspectratio}
% Set default figure placement to htbp
\makeatletter
\def\fps@figure{htbp}
\makeatother

\KOMAoption{captions}{tableheading}
\makeatletter
\@ifpackageloaded{caption}{}{\usepackage{caption}}
\AtBeginDocument{%
\ifdefined\contentsname
  \renewcommand*\contentsname{Table of contents}
\else
  \newcommand\contentsname{Table of contents}
\fi
\ifdefined\listfigurename
  \renewcommand*\listfigurename{List of Figures}
\else
  \newcommand\listfigurename{List of Figures}
\fi
\ifdefined\listtablename
  \renewcommand*\listtablename{List of Tables}
\else
  \newcommand\listtablename{List of Tables}
\fi
\ifdefined\figurename
  \renewcommand*\figurename{Figure}
\else
  \newcommand\figurename{Figure}
\fi
\ifdefined\tablename
  \renewcommand*\tablename{Table}
\else
  \newcommand\tablename{Table}
\fi
}
\@ifpackageloaded{float}{}{\usepackage{float}}
\floatstyle{ruled}
\@ifundefined{c@chapter}{\newfloat{codelisting}{h}{lop}}{\newfloat{codelisting}{h}{lop}[chapter]}
\floatname{codelisting}{Listing}
\newcommand*\listoflistings{\listof{codelisting}{List of Listings}}
\makeatother
\makeatletter
\makeatother
\makeatletter
\@ifpackageloaded{caption}{}{\usepackage{caption}}
\@ifpackageloaded{subcaption}{}{\usepackage{subcaption}}
\makeatother
\ifLuaTeX
  \usepackage{selnolig}  % disable illegal ligatures
\fi
\usepackage{bookmark}

\IfFileExists{xurl.sty}{\usepackage{xurl}}{} % add URL line breaks if available
\urlstyle{same} % disable monospaced font for URLs
\hypersetup{
  pdftitle={Fisheries Officers Quantitative Analysis},
  colorlinks=true,
  linkcolor={blue},
  filecolor={Maroon},
  citecolor={Blue},
  urlcolor={Blue},
  pdfcreator={LaTeX via pandoc}}

\title{Fisheries Officers Quantitative Analysis}
\author{}
\date{}

\begin{document}
\maketitle

\subsection{Role of officers
interviewed}\label{role-of-officers-interviewed}

The illustration of Table~\ref{tbl-role} indicates that 73\% of the
surveyed government officers were fisheries officers while only 27\%
were extension officers. This meant that most of the survey government
officers were located in stations where farmers popped in seeking
aquaculture information. Extension officers were very few thereby
indicating a low level of extension services in the aquaculture sector.

\begin{longtable}[]{@{}lc@{}}

\caption{\label{tbl-role}The Role of government officers interviewed}

\tabularnewline

\toprule\noalign{}
\textbf{Characteristic} & \textbf{N = 11} \\
\midrule\noalign{}
\endhead
\bottomrule\noalign{}
\endlastfoot
Role & \\
Extension officer & 3 (27\%) \\
Fisheries officer & 8 (73\%) \\

\end{longtable}

\subsection{Aquaculture Officers Education
Level}\label{aquaculture-officers-education-level}

Majority of the officers held a bachelor's degree (82\%) with some
holding masters degrees and diplomas. The results showed that most of
the employed officers had an undergradaute education thereby indicating
a significant level of education.

\begin{longtable}[]{@{}lc@{}}

\caption{\label{tbl-edulvl}Education of Level of Government officers in
the Aquaculture Sector}

\tabularnewline

\toprule\noalign{}
\textbf{Characteristic} & \textbf{N = 11} \\
\midrule\noalign{}
\endhead
\bottomrule\noalign{}
\endlastfoot
education level & \\
Bachelor's Degree & 9 (82\%) \\
Diploma & 1 (9.1\%) \\
Master's degree & 1 (9.1\%) \\

\end{longtable}

\subsection{Formal Training in Fish
Health}\label{formal-training-in-fish-health}

A considerable proportion of officers had received formal training on
fish health that included subjects such as fish pathology, parasitology
and general fish hygiene.

\begin{longtable}[]{@{}lc@{}}

\caption{\label{tbl-fhealth}Formal fish health training}

\tabularnewline

\toprule\noalign{}
\textbf{Characteristic} & \textbf{N = 11} \\
\midrule\noalign{}
\endhead
\bottomrule\noalign{}
\endlastfoot
formal fish health training & \\
No & 4 (36\%) \\
Yes & 7 (64\%) \\

\end{longtable}

\subsection{Knowledge on Common fish
diseases}\label{knowledge-on-common-fish-diseases}

A considerable proportion of officers had somewhatfamiliar knowledge on
common fish diseases (55\%) while only 36\% were very familiar with
common fish diseases. Interestingly 9\% of the officers were not familar
at all with common fish diseases.

\begin{longtable}[]{@{}lc@{}}

\caption{\label{tbl-diseases}Common Fish Diseases Knowledge}

\tabularnewline

\toprule\noalign{}
\textbf{Characteristic} & \textbf{N = 11} \\
\midrule\noalign{}
\endhead
\bottomrule\noalign{}
\endlastfoot
common fish diseases knowledge & \\
Not at all familiar & 1 (9.1\%) \\
Somewhat familiar & 6 (55\%) \\
Very familiar & 4 (36\%) \\

\end{longtable}

\subsection{Parasites}\label{parasites}

The fish parasites shown in Table~\ref{tbl-parasites} were the parasites
that most of the fisheries could easily identify because they were
familiar to them.

\begin{longtable}[]{@{}lc@{}}

\caption{\label{tbl-parasites}Common Fish Parasites}

\tabularnewline

\toprule\noalign{}
\textbf{Characteristic} & \textbf{N = 20} \\
\midrule\noalign{}
\endhead
\bottomrule\noalign{}
\endlastfoot
common parasites & \\
& 1 (5.0\%) \\
Bacteria & 1 (5.0\%) \\
Cestodes & 1 (5.0\%) \\
Clinostomum & 1 (5.0\%) \\
Dactylogyrus & 1 (5.0\%) \\
Ichthyobo & 1 (5.0\%) \\
Leeches & 1 (5.0\%) \\
Round worms & 1 (5.0\%) \\
Tape worms & 1 (5.0\%) \\
Tapeworm & 1 (5.0\%) \\
1. Acathocephalans & 1 (5.0\%) \\
Flukes & 1 (5.0\%) \\
Fungus & 1 (5.0\%) \\
Gyro dactyls & 1 (5.0\%) \\
Leaches & 1 (5.0\%) \\
Nematodes & 1 (5.0\%) \\
Protozoas & 1 (5.0\%) \\
Round worm & 1 (5.0\%) \\
Tape worm & 1 (5.0\%) \\
Uronema & 1 (5.0\%) \\

\end{longtable}

\subsection{Implementation of
Biosecurity}\label{implementation-of-biosecurity}

The implementation of biosecurity measures and plans are not among the
subjects that most officers had knowledge about because 60\% of the
interviewed officers only had moderate knowledge 10\% were not aware at
all while only 30\% had enough knoledge on biosecurity measures as
illustrated in Table~\ref{tbl-biosecurity}.

\begin{longtable}[]{@{}lc@{}}

\caption{\label{tbl-biosecurity}Knowledge of the Implementation of
Biosecurity}

\tabularnewline

\toprule\noalign{}
\textbf{Characteristic} & \textbf{N = 11} \\
\midrule\noalign{}
\endhead
\bottomrule\noalign{}
\endlastfoot
biosecurity implementation & \\
Highly knowledgeable & 3 (30\%) \\
Moderately knowledgeable & 6 (60\%) \\
Not knowledgeable & 1 (10\%) \\

\end{longtable}

\subsection{Tools and Techniques of identifying fish diseases and health
issues}\label{tools-and-techniques-of-identifying-fish-diseases-and-health-issues}

As illustrated in Table~\ref{tbl-tools} 90\% were of officers
interviewed used clinical observation or simply observation as a tool of
identifying fish health related issues. In addition only 40\% used
microscopic examination to investigate fish health concerns while only
20\% ever used PCR or any molecular testing methods.

\begin{longtable}[]{@{}lc@{}}

\caption{\label{tbl-tools}Tools and Tecchniques}

\tabularnewline

\toprule\noalign{}
\textbf{Characteristic} & \textbf{N = 11} \\
\midrule\noalign{}
\endhead
\bottomrule\noalign{}
\endlastfoot
Microscopic examination & 4 (40\%) \\
PCR testing & 2 (20\%) \\
Clinical observation & 9 (90\%) \\

\end{longtable}

\subsection{Frequency of
consultations}\label{frequency-of-consultations}

The survey sought to determine the frequency at which the officers were
consulted by fish farmes. As illustrated in Table~\ref{tbl-consult},
most officers indicated that they were consulted occassionally to
provide information on fish health while 40\% were frequently consulted
while very few 10\% were rarely consulted. This showed that farmers were
actively seeking information from government establishments.

\begin{longtable}[]{@{}lc@{}}

\caption{\label{tbl-consult}Consultations from Fish Farmers}

\tabularnewline

\toprule\noalign{}
\textbf{Characteristic} & \textbf{N = 11} \\
\midrule\noalign{}
\endhead
\bottomrule\noalign{}
\endlastfoot
consultation frequency & \\
Frequently & 4 (40\%) \\
Occassionaly & 5 (50\%) \\
Rarely & 1 (10\%) \\

\end{longtable}

\subsection{Dissemination of Fish Health
Information}\label{dissemination-of-fish-health-information}

The Kenya Aquaculture space is inundated with worshops and seminars and
according to Table~\ref{tbl-disse} most dissemination of fish health
information takes place during workshops and seminars (67\%), followed
by online platforms (33\%) and lastly on printed materials (11\%) which
are rare in the public domain. Other sources of information are during
farm visits and farmer to farmer exchange of information and finally in
organised farmer group trainings and meetings. There are usually
collaborations between departments and agencies to enhance the sharing
of information.

\begin{longtable}[]{@{}lc@{}}

\caption{\label{tbl-disse}Dissemination of Information}

\tabularnewline

\toprule\noalign{}
\textbf{Characteristic} & \textbf{N = 11} \\
\midrule\noalign{}
\endhead
\bottomrule\noalign{}
\endlastfoot
Workshops and seminars & 6 (67\%) \\
Printed materials & 1 (11\%) \\
Online platforms & 3 (33\%) \\

\end{longtable}

\subsection{Challenges in mitigating the effects of fish diseases and
parasites}\label{challenges-in-mitigating-the-effects-of-fish-diseases-and-parasites}

The challenges faced by government officers are illustrated in
Table~\ref{tbl-challenge} below.

\begin{longtable}[]{@{}
  >{\raggedright\arraybackslash}p{(\columnwidth - 2\tabcolsep) * \real{0.8919}}
  >{\centering\arraybackslash}p{(\columnwidth - 2\tabcolsep) * \real{0.1081}}@{}}

\caption{\label{tbl-challenge}Challenges}

\tabularnewline

\toprule\noalign{}
\begin{minipage}[b]{\linewidth}\raggedright
\textbf{Characteristic}
\end{minipage} & \begin{minipage}[b]{\linewidth}\centering
\textbf{N = 15}
\end{minipage} \\
\midrule\noalign{}
\endhead
\bottomrule\noalign{}
\endlastfoot
challenges & \\
cestodes & 1 (6.7\%) \\
insufficient funding on movement to farmers/ extension services & 1
(6.7\%) \\
nematodes & 1 (6.7\%) \\
Acathocephalans & 1 (6.7\%) \\
cost of treatment & 1 (6.7\%) \\
Few laboratories and lack of analysis/ testing equipment & 1 (6.7\%) \\
Inadequate materials of reference & 1 (6.7\%) \\
Lack diagnostic tools and reagents and a specialized lab to ascertain
occurrences of the parasites & 1 (6.7\%) \\
Lack of lab services nearby & 1 (6.7\%) \\
Lack of sample collection and testing tools & 1 (6.7\%) \\
Language understanding & 1 (6.7\%) \\
methods of specific diseases control & 1 (6.7\%) \\
NA & 1 (6.7\%) \\
Reaching to them is a challenge & 1 (6.7\%) \\
Some are asking for feeds and ponds cleaning facilitation from our side
& 1 (6.7\%) \\

\end{longtable}

\subsection{Specific areas of fish health
management}\label{specific-areas-of-fish-health-management}

The officers indicated interest in learning common and advanced diseases
identification tools and techniques and methods. Information on disease
management would be valuable to them and prevention of diases, pests and
parasites.

\subsection{Feedback from Farmers}\label{feedback-from-farmers}

Feedback from farmers majorly is verbal when officers meet in farms
during research or extension excursions.

\subsection{Methods of evaluating success of
interverntions}\label{methods-of-evaluating-success-of-interverntions}

Methods of monitoring and evalauting the success of interventions on
fish health are illustrated in Table~\ref{tbl-success} below.

\begin{longtable}[]{@{}lc@{}}

\caption{\label{tbl-success}Assessment of the Success of Interventions}

\tabularnewline

\toprule\noalign{}
\textbf{Characteristic} & \textbf{N = 11} \\
\midrule\noalign{}
\endhead
\bottomrule\noalign{}
\endlastfoot
interventions success assessment & \\
estimation of the affected fish verses the healthy ones & 1 (10\%) \\
Feedback during field visit or on phone & 1 (10\%) \\
Great improvement especially in management of fish farms & 1 (10\%) \\
Making frequent farm visit & 1 (10\%) \\
NA & 1 (10\%) \\
Not quantitave results can be observed & 1 (10\%) \\
Reduced number of pond fish kills & 1 (10\%) \\
The management level is average & 1 (10\%) \\
Well being of the fish & 1 (10\%) \\
When the affected fish regain normalcy & 1 (10\%) \\

\end{longtable}



\end{document}
